%==============================================================================
% Sjabloon onderzoeksvoorstel bachproef
%==============================================================================
% Gebaseerd op document class `hogent-article'
% zie <https://github.com/HoGentTIN/latex-hogent-article>
% Voor een voorstel in het Engels: voeg de documentclass-optie [english] toe.
% Let op: kan enkel na toestemming van de bachelorproefcoördinator!
\documentclass{hogent-article}

% Invoegen bibliografiebestand
\addbibresource{voorstel.bib}

% Informatie over de opleiding, het vak en soort opdracht
\studyprogramme{Professionele bachelor toegepaste informatica}
\course{Bachelorproef}
\assignmenttype{Onderzoeksvoorstel}
% Voor een voorstel in het Engels, haal de volgende 3 regels uit commentaar
% \studyprogramme{Bachelor of applied information technology}
% \course{Bachelor thesis}
% \assignmenttype{Research proposal}

\academicyear{2025-2026}

\title{Optimalisatie van het beheer van leeggoedpalletten bij Freshfrozen For Pets: Een onderzoek naar een digitale oplossingen voor foutenreductie en gebruiksvriendelijkheid.}

\author{Wout Schelstraete}
\email{wout.schelstraete@student.hogent.be}
\projectrepo{https://github.com/WoutSHOGENT/Bachelorproef}

\supervisor[Co-promotor]{P. Kemseke (Freshfrozen, \href{mailto:Pierre.kemseke@jolipet.com}{Pierre.kemseke@jolipet.com})}

% Binnen welke specialisatierichting uit 3TI situeert dit onderzoek zich?
% Kies uit deze lijst:
%
% - Mobile \& Enterprise development
% - AI \& Data Engineering
% - Functional \& Business Analysis
% - System \& Network Administrator
% - Mainframe Expert
% - Als het onderzoek niet past binnen een van deze domeinen specifieer je deze
%   zelf
%
\specialisation{Full-Stack Developer}
\keywords{Stockbeheer, ERP, API, Middleware}

\begin{document}

\begin{abstract} 
  Dit voorstel richt zich op moeilijkheden bij het beheer van leeggoedpalletten bij Freshfrozen For Pets. Zij gebruiken op dit moment gebruik van meervoudige Excel-spreadsheets. Het doel van dit voorstel is om te onderzoeken hoe het beheer van efficiënter kan worden georganiseerd. De methodologie omvat een evaluatie van drie verschillende oplossingen: directe implementatie in het huidige ERP-systeem, integratie via API-calls, en de bouw van een standalone applicatie. de technische en budgetaire haalbaarheid van elke oplossing wordt per case bekeken. Op die manier komen we voor elke oplossing op een proof-of-concept. De verwachting is dat een API-integratie de meest waardevolle oplossing zal zijn, maar dat een standalone applicatie het meest haalbaar is om het proces voor de productiechef en de financieel bedienden verbeteren. 
\end{abstract}

\tableofcontents

% De hoofdtekst van het voorstel zit in een apart bestand, zodat het makkelijk
% kan opgenomen worden in de bijlagen van de bachelorproef zelf.
%---------- Inleiding ---------------------------------------------------------
\section{Inleiding}%
\label{sec:inleiding}

Freshfrozen werkt op dit moment nog veel met Excel. Specifiek voor dit voorstel gaat het over het gebruik van Excel voor stockbeheer van Leeggoedpalletten. Freshfrozen krijgt grondstoffen die gebruikt worden voor productie van dierenvoeding op deze Leeggoedpalletten. Zo ontstaat per klant een Excel file waar bijgehouden wordt hoeveel ze binnenkrijgen en hoeveel ze terugsturen. Hier kruipen op termijn fouten in en kan leiden tot grote verschillen. Deze Excels worden ingevuld door de verantwoordelijke van de productielijn. Daarna worden ze gecontroleerd door een bediende binnen het bedrijf. Aan beide kanten kunnen fouten gebeuren. De onderzoeksvraag luid dus als volgt: Hoe kan het beheer van leeggoedpalletten bij Freshfrozen For Pets efficiënter georganiseerd worden met als focus om tijdbesparing, foutenreductie, gebruiksvriendelijkheid en overzichtelijkheid te verbeteren?

Het doel is om een proof-of-concept te maken van een robuuste applicatie die integreert met het huidige ERP-systeem van Freshfrozen. Ik wil onderzoeken of het mogelijk is om in een bestaand ERP-systeem het mogelijk is om zelf een integratie te schrijven. Ik let daarbij op budget, benodigde tijd en of ik van de fabrikant zelf een integratie mag doen. Ik zie mijn bachelorproef als succes als ik opzenminst deze 3 routes heb geprobeerd om dit probleem op te lossen:

\begin{itemize}
  \item Directe implementatie in het bestaande ERP-systeem
  \item Een eigen applicatie te integreren in het systeem door middel van API-calls
  \item Een standalone applicatie te bouwen
\end{itemize}

%---------- Stand van zaken ---------------------------------------------------

\section{Literatuurstudie}%
\label{sec:literatuurstudie}

Hier beschrijf je de \emph{state-of-the-art} rondom je gekozen onderzoeksdomein, d.w.z.\ een inleidende, doorlopende tekst over het onderzoeksdomein van je bachelorproef. Je steunt daarbij heel sterk op de professionele \emph{vakliteratuur}, en niet zozeer op populariserende teksten voor een breed publiek. Wat is de huidige stand van zaken in dit domein, en wat zijn nog eventuele open vragen (die misschien de aanleiding waren tot je onderzoeksvraag!)?

Je mag de titel van deze sectie ook aanpassen (literatuurstudie, stand van zaken, enz.). Zijn er al gelijkaardige onderzoeken gevoerd? Wat concluderen ze? Wat is het verschil met jouw onderzoek?

Verwijs bij elke introductie van een term of bewering over het domein naar de vakliteratuur, bijvoorbeeld~\autocite{Hykes2013}! Denk zeker goed na welke werken je refereert en waarom.

Draag zorg voor correcte literatuurverwijzingen! Een bronvermelding hoort thuis \emph{binnen} de zin waar je je op die bron baseert, dus niet er buiten! Maak meteen een verwijzing als je gebruik maakt van een bron. Doe dit dus \emph{niet} aan het einde van een lange paragraaf. Baseer nooit teveel aansluitende tekst op eenzelfde bron.

Als je informatie over bronnen verzamelt in JabRef, zorg er dan voor dat alle nodige info aanwezig is om de bron terug te vinden (zoals uitvoerig besproken in de lessen Research Methods).

% Voor literatuurverwijzingen zijn er twee belangrijke commando's:
% \autocite{KEY} => (Auteur, jaartal) Gebruik dit als de naam van de auteur
%   geen onderdeel is van de zin.
% \textcite{KEY} => Auteur (jaartal)  Gebruik dit als de auteursnaam wel een
%   functie heeft in de zin (bv. ``Uit onderzoek door Doll & Hill (1954) bleek
%   ...'')

Je mag deze sectie nog verder onderverdelen in subsecties als dit de structuur van de tekst kan verduidelijken.

%---------- Methodologie ------------------------------------------------------
\section{Methodologie}%
\label{sec:methodologie}
Zoals ik in de inleiding vermelde wil ik 3 luiken gaan onderzoeken/uitvoeren. Daardoor zal deze bachelorproef verlopen in 3 fases. 
\subsection{Fase 1 - Implementatie in het ERP-systeem}
In deze fase kijk ik hoe het huidige ERP-systeem in elkaar zit. Hieronder valt onderzoek doen naar wat het systeem precies is en contact opnemen met het bedrijf. Op die manier weet ik of ik met deze fase verder kan. Het doel in deze fase is een proof-of-concept maken dat directe implementatie in een bestaand ERP-systeem mogelijk is. Ook kijk ik naar de tijd dat dit zou kosten maar ook budget. De proof-of-concept bevat een duidelijk voorbeeld van een zelf geïmplementeerd systeem in de vorm van data die doorgegeven wordt binnen het systeem. In deze fase zal ik mij ook moeten aanpassen aan welke tools er op dit moment gebruikt worden om het ERP-systeem te bouwen. Afhankelijk van de tools en de mogelijkheid tot directe implementatie kan deze fase langer of korter duren. In het geval van geen mogelijke implentatie zal deze fase lopen over een termijn van 5 dagen. In het geval van mogelijke implentatie zal deze fase lopen over een termijn van 15 dagen.
\subsection{Fase 2 - Communicatie met het ERP-Systeem}
Hier onderzoek ik of communicatie met het ERP-Systeem een optie is. Dit door middel van API-calls die van en naar het systeem worden gestuurd. Ook hier wordt gekeken naar tijd en budget. Het doel van deze fase is om opnieuw een proof-of-concept te maken. Deze keer bevat deze een duidelijk voorbeeld van communicatie tussen een eigen uitwerking en het ERP-systeem. Zo kan het ERP-systeem API-calls gebruiken om op een externe database CRUD operations te doen. Ook de mogelijkheid om dit te doen moet onderzocht en besproken worden met de leverancier van het ERP-systeem.



Hier beschrijf je hoe je van plan bent het onderzoek te voeren. Welke onderzoekstechniek ga je toepassen om elk van je onderzoeksvragen te beantwoorden? Gebruik je hiervoor literatuurstudie, interviews met belanghebbenden (bv.~voor requirements-analyse), experimenten, simulaties, vergelijkende studie, risico-analyse, PoC, \ldots?

Valt je onderwerp onder één van de typische soorten bachelorproeven die besproken zijn in de lessen Research Methods (bv.\ vergelijkende studie of risico-analyse)? Zorg er dan ook voor dat we duidelijk de verschillende stappen terug vinden die we verwachten in dit soort onderzoek!

Vermijd onderzoekstechnieken die geen objectieve, meetbare resultaten kunnen opleveren. Enquêtes, bijvoorbeeld, zijn voor een bachelorproef informatica meestal \textbf{niet geschikt}. De antwoorden zijn eerder meningen dan feiten en in de praktijk blijkt het ook bijzonder moeilijk om voldoende respondenten te vinden. Studenten die een enquête willen voeren, hebben meestal ook geen goede definitie van de populatie, waardoor ook niet kan aangetoond worden dat eventuele resultaten representatief zijn.

Uit dit onderdeel moet duidelijk naar voor komen dat je bachelorproef ook technisch voldoen\-de diepgang zal bevatten. Het zou niet kloppen als een bachelorproef informatica ook door bv.\ een student marketing zou kunnen uitgevoerd worden.

Je beschrijft ook al welke tools (hardware, software, diensten, \ldots) je denkt hiervoor te gebruiken of te ontwikkelen.

Probeer ook een tijdschatting te maken. Hoe lang zal je met elke fase van je onderzoek bezig zijn en wat zijn de concrete \emph{deliverables} in elke fase?

%---------- Verwachte resultaten ----------------------------------------------
\section{Verwacht resultaat, conclusie}%
\label{sec:verwachte_resultaten}

Hier beschrijf je welke resultaten je verwacht. Als je metingen en simulaties uitvoert, kan je hier al mock-ups maken van de grafieken samen met de verwachte conclusies. Benoem zeker al je assen en de onderdelen van de grafiek die je gaat gebruiken. Dit zorgt ervoor dat je concreet weet welk soort data je moet verzamelen en hoe je die moet meten.

Wat heeft de doelgroep van je onderzoek aan het resultaat? Op welke manier zorgt jouw bachelorproef voor een meerwaarde?

Hier beschrijf je wat je verwacht uit je onderzoek, met de motivatie waarom. Het is \textbf{niet} erg indien uit je onderzoek andere resultaten en conclusies vloeien dan dat je hier beschrijft: het is dan juist interessant om te onderzoeken waarom jouw hypothesen niet overeenkomen met de resultaten.



\printbibliography[heading=bibintoc]

\end{document}