%---------- Inleiding ---------------------------------------------------------
\section{Inleiding}%
\label{sec:inleiding}

\textcolor{magenta}{Freshfrozen werkt op dit moment nog veel met Excel. in deze specifieke casus gaat het over het gebruik van Excel voor stockbeheer van leeggoedpalletten. Freshfrozen krijgt grondstoffen die gebruikt worden voor productie van dierenvoeding op deze leeggoedpalletten binnen. Zo ontstaat per klant een Excel bestand waar bijgehouden wordt hoeveel palletten ze binnenkrijgen en hoeveel ze er terugsturen. Hier kruipen op termijn fouten in en dit kan leiden tot grote verschillen. Deze Excel spreadsheets worden ingevuld door de Productiechef. Daarna worden ze gecontroleerd door een financieel bediende binnen het bedrijf. Bij beide gebruikers kunnen fouten ontstaan.} 
Dit is een zeer interessant probleem. Een betere oplossing is zeker mogelijk. Het doel van deze bachelorproef is om dit probleem op te lossen. De onderzoeksvraag luidt dus als volgt: \textcolor{blue}{Hoe kan het beheer van leeggoedpalletten bij Freshfrozen For Pets efficiënter georganiseerd worden met als focus om tijdbesparing, foutenreductie, gebruiksvriendelijkheid en overzichtelijkheid te verbeteren?} 
Daarnaast zijn ook enkele deelvragen opgesteld. Deze kunnen worden verdeeld in twee groepen: het probleem- en oplossingsdomein. De deelvragen omtrent het probleemdomein gaan als volgt: \textcolor{magenta}{Hoe verloopt het beheer van leeggoedpalletten binnen Freshfrozen For Pets op dit moment, en waar treden de meeste inefficiënties of fouten op?; Hoe ervaren de productiechef en financieel bedienden het systeem op dit moment, en wat zijn voor hun de grootste voor- en nadelen?} Volgende vragen betrekken zich tot het oplossingsdomein:  \textcolor{blue}{Wat zijn 3 concrete voor- en nadelen van een centrale technische oplossing tegenover het gebruik van Excel voor overzicht?; Kan er direct geïmplenteerd worden in het huidige ERP systeem van Mapa dat gebruikt wordt door Freshfrozen?; Met welke standaarden werken andere bedrijven?; Welke 2 verbeteringen zijn van het grootste belang voor Freshfrozen en leveren dus de grootste klantenwaarde?}


Het doel is om een proof-of-concept te maken van een robuuste applicatie die integreert met het huidige ERP-systeem van Freshfrozen. Er wordt onderzocht of het mogelijk is om in een bestaand ERP-systeem zelf een integratie te schrijven. Daarbij wordt gelet op budget, benodigde tijd en of van de fabrikant zelf een integratie mag worden gemaakt. Deze bachelorproef wordt gezien als succesvol als volgende 3 routes geprobeerd zijn om dit probleem op te lossen:

\begin{itemize}
  \item Directe implementatie in het bestaande ERP-systeem
  \item Een eigen applicatie te integreren in het systeem door middel van API-calls
  \item Een standalone applicatie te bouwen
\end{itemize}

%---------- Stand van zaken ---------------------------------------------------
\section{Literatuurstudie}%
\label{sec:literatuurstudie}

Excel wordt doorgaans nog veel gebruikt door bedrijven. Het is een gemakkelijke optie om data op te slaan. Het probleem vormt zich echter wanneer er meerdere bestanden ontstaan. Zo kan data voor een bepaald bestand ergens andere terecht komen. Hierbij komt geen validatie te pas, wat zorgt voor een grote kans op menselijke fouten. 
\subsection{De nadelen van Microsoft Excel}
Volgens een studie door \textcite{Aburas2019} zijn er drie types fouten die in een spreadsheet kunnen sluipen. Dit zijn: Mechanical errors, Omission errors en Logic errors. Mechanical errors zijn simpele fouten die gaan over het ingeven van nummers en referenties. Logic errors komen voor wanneer een verkeerde formule gegeven wordt aan een bepaalde cel. Als een formule verkeerd is is dit moeilijk terug te vinden als je niet alles weet van het betreffende vakgebied. Deze twee fouten zijn degene die voorkomen in het geval van Freshfrozen. Weten wat de aard is van deze fouten, en begrijpen wat ze zijn is heel belangrijk om het probleem te kunnen oplossen.
Het concrete probleem met spreadsheets is dat data niet abstract is. de data die je invoert en de berekeningen bevinden zich in hetzelfde bestand \autocite{Aburas2019}. Verder beschrijft \textcite{Fuller2011} dat de intentie om simpele data op te slaan snel kan evolueren in een complex systeem waar fouten snel tevoorschijn komen. Dit volgt dezelfde conclusie van daarnet. De data is niet abstract genoeg, wat zorgt voor complexe bestanden waar snel fouten in kruipen.
\subsection{ERP als oplossing}
Een ERP of Enterprise Resource Planning is een vorm van software gebruikt door veel bedrijven. Een ERP systeem dient als digitale schakel bij bedrijfsprocessen. Dit zorgt ervoor dat deze processen te vereenvoudigen en versnellen over meerdere departementen van een bedrijf heen \autocite{11162534}. Freshfrozen gebruikt op dit moment al een ERP pakket, gemaakt door Mapa. Het stockbeheer van de leeggoedpalletten zit op dit moment nog niet in dit pakket. Een ideale oplossing zou een point-to-point integratie zijn. Dit is een aangepaste implementatie waarbij 2 verschillende producten in direct contact met elkaar staan. dit voldoet zeker aan alle noden van de business, maar is duur, tijdsrovend en moeilijk te onderhouden \autocite{Sah2025}. Als er na deze optie geen mogelijkheid is tot direct implenteren in het pakket zelf is er ook nog een derde optie: Een middleware solution. Dit is een systeem dat zich tussen het eigen systeem en het ERP systeem bevind. Dit reduceert de complexiteit van directe interactie tussen de twee systemen, maar is wel zelf complex om te implementeren en vereist ook continu onderhoud \autocite{Sah2025}. Deze informatie helpt ons op dit moment niet verder om te weten wat de beste optie is. Verder eigen onderzoek is dus vereist.
\subsection{Een eigen applicatie als oplossing}
Dit is de laatste optie. Een aparte applicatie geeft ons complete vrijheid over wat er geïmplementeerd wordt en op welke manier. Volgens een projectrapport door \textcite{MACHADO2023184} is .NET en Blazor WebAssembly een goede optie om een beheerapplicatie te maken. Dit komt door de voordelen op vlak van beveiliging en snelheid. Dit is ook de goedkoopste optie. Volgens een artikel door \textcite{machines13111002} wordt een aangepaste digitale oplossing zelfs aangeraden voor kmo's. zelfs het herwerken van een klein bedrijfsproces kan een katalysator zijn voor digitale maturiteit binnen een bedrijf.
% Voor literatuurverwijzingen zijn er twee belangrijke commando's:
% \autocite{KEY} => (Auteur, jaartal) Gebruik dit als de naam van de auteur
%   geen onderdeel is van de zin.
% \textcite{KEY} => Auteur (jaartal)  Gebruik dit als de auteursnaam wel een
%   functie heeft in de zin (bv. ``Uit onderzoek door Doll & Hill (1954) bleek
%   ...'')

%---------- Methodologie ------------------------------------------------------
\section{Methodologie}%
\label{sec:methodologie}
Zoals reeds in de inleiding vermeld, worden 3 luiken onderzocht/uitgevoerd. Daardoor zal deze bachelorproef verlopen in 3 fases.
\subsection{Fase 0 - Intern onderzoek}
Voor er begonnen kan worden aan het implementeren van de applicatie, moet er eerst goed begrepen worden op welke manier het probleem zich uit en wat eraan te doen is. Hiervoor wordt contact opgenomen met beide de productiechef en financieel bedienden binnen Freshfrozen. Zij bieden een goed antwoord op de deelvragen omtrent het probleemdomein doordat zij zelf zo dicht bij het probleem staan. Verder wordt gevraagd wat voor P. Kemseke (CEO van FreshFrozen) de 2 belangrijkste verbeteringen zouden zijn. Zo is er duidelijkheid over waar voor hem de focus moet liggen. Nadat al deze vragen gesteld zijn kan er pas echt van start gegaan worden.
\subsection{Fase 1 - Implementatie in het ERP-systeem}
In deze fase wordt er bekeken hoe het huidige ERP-systeem in elkaar zit. Hieronder valt onderzoek naar wat het systeem precies is en contact opnemen met het bedrijf die de software produceerde. Op die manier is er genoeg informatie om te weten of er met deze fase verder gewerkt kan worden. Het doel in deze fase is een proof-of-concept maken dat directe implementatie in een bestaand ERP-systeem mogelijk is. Ook wordt er gekeken naar de tijd dat dit zou kosten maar ook budget. De proof-of-concept bevat een duidelijk voorbeeld van een zelf geïmplementeerd systeem in de vorm van data die doorgegeven wordt binnen het systeem. In deze fase wordt er ook aangepast aan de tools die op dit moment gebruikt worden om het ERP-systeem te bouwen. Afhankelijk van de tools en de mogelijkheid tot directe implementatie kan deze fase langer of korter duren. In het geval van geen mogelijke implementatie zal deze fase lopen over een termijn van 5 dagen. In het geval van mogelijke implementatie zal deze fase lopen over een termijn van 10 dagen.
\subsection{Fase 2 - Communicatie met het ERP-Systeem}
Hier wordt onderzocht of communicatie met het ERP-Systeem een optie is. Dit door middel van API-calls die van en naar het systeem worden gestuurd. Ook hier wordt gekeken naar tijd en budget. Het doel van deze fase is om opnieuw een proof-of-concept te maken. Deze keer bevat deze een duidelijk voorbeeld van communicatie tussen een eigen uitwerking en het ERP-systeem. Zo kan het ERP-systeem API-calls gebruiken om op een externe database CRUD operations te doen. Ook de mogelijkheid om dit te doen moet onderzocht en besproken worden met de leverancier van het ERP-systeem. Voor het uitwerken van deze fase wordt gebruik gemaakt van een .NET backend die werkt als basis voor de CRUD operations. Dit wordt ook het aanspreekpunt van het ERP-systeem om data op te halen door middel van API-calls. Deze fase loopt over een termijn van 10 dagen.
\subsection{Fase 3 - Standalone applicatie}
Na het uitwerken van de vorige 2 fases volgt de implementatie van een standalone uitwerking die het probleem van Freshfrozen oplost. Het budget wordt hierbij opnieuw onderzocht. Het doel van deze fase is een standalone applicatie uitwerken die aan alle criteria van de onderzoeksvraag voldoet. Deze zal gebruik maken van de industriestandaard die terug te vinden is in de literatuurstudie. Deze fase zal lopen over een termijn van 15 dagen. 

%---------- Verwachte resultaten ----------------------------------------------
\section{Verwacht resultaat, conclusie}%
\label{sec:verwachte_resultaten}
De verwachting is dat de eerste fase niet mogelijk zal zijn. Dit door het feit dat het ERP systeem van Mapa closed-source is en de licentie niet toestaat dat de software handmatig aangepast wordt. Verder wordt verwacht dat fase twee mogelijk is, en ook de makkelijkste oplossing is voor Freshfrozen zelf. Zo moeten ze zich niet aanpassen aan een volledig nieuwe applicatie. De laatste verwachting is dat fase drie zeker mogelijk is en ook de meest rechtstreekse oplossing zal bieden. Zowel de Productiechef als de financieel bediende binnen Freshfrozen ervaren minder tijdverlies en minder kosten door menselijke fouten.
